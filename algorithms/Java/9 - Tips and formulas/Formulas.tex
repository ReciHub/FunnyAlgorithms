\begin{center}
\tablefirsthead{}
\tabletail{
\midrule 
\multicolumn{2}{r}{{Continúa en la siguiente columna}} \\}
\tablelasttail{}
{\renewcommand{\arraystretch}{1.4}
\begin{supertabular}{|p{2.2cm}|p{8.2cm}|}
\hline
\multicolumn{2}{|c|}{} \\
\multicolumn{2}{|c|}{PERMUTACIÓN Y COMBINACIÓN} \\
\multicolumn{2}{|c|}{} \\ \hline
Combinación (Coeficiente Binomial) & Número de subconjuntos de k elementos escogidos de un conjunto con n elementos.

$ \binom{n}{k} = \binom{n}{n-k} = \displaystyle\frac{n!}{k!(n-k)!} $ 

\\ \hline

Combinación con repetición & Número de grupos formados por n elementos, partiendo de m tipos de elementos.

$ CR_{m}^{n} = \binom{m+n-1}{n} = \displaystyle\frac{(m + n - 1)!}{n!(m-1)!} $

\\ \hline
Permutación & Número de formas de agrupar n elementos, donde importa el orden y sin repetir elementos

$ P_{n} = n! $
\\ \hline
Permutación múltiple & 
Elegir r elementos de n posibles con repetición 


$ n^{r} $
\\ \hline
Permutación con repetición & Se tienen n elementos donde el primer elemento se repite a veces , el segundo b veces , el tercero c veces, ...

$ PR_{n}^{a,b,c...} = \displaystyle\frac{P_{n}}{a!b!c!...}$

\\ \hline
Permutaciones sin repetición & Núumero de formas de agrupar r elementos de n disponibles, sin repetir elementos


$\displaystyle\frac{n!}{(n-r)!}$

\\ \hline
\multicolumn{2}{|c|}{} \\
\multicolumn{2}{|c|}{DISTANCIAS} \\
\multicolumn{2}{|c|}{} \\ \hline
Distancia Euclideana & $d_{E}(P_{1},P_{2}) = \sqrt{(x_{2}-x_{1})^{2}+(y_{2}-y_{1})^{2}}$ \\ \hline
Distancia Manhattan & $d_{M}(P_{1}, P_{2}) = |x_{2} - x_{1}| + |y_{2} - y_{1}|$ \\ \hline
\multicolumn{2}{|c|}{} \\
\multicolumn{2}{|c|}{CIRCUNFERENCIA Y CÍRCULO} \\ 
\multicolumn{2}{|c|}{} \\ \hline
\multicolumn{2}{|p{12cm}|}{Considerando $r$ como el radio, $\alpha$ como el ángulo del arco o sector, y (R, r) como radio mayor y menor respectivamente.} \\ \hline
Área                   & $A = \pi * r^{2} $\\ \hline
Longitud               & $L = 2*\pi*r$  \\ \hline
Longitud de un arco    & $L = \displaystyle\frac{2*\pi*r*\alpha}{360}$  \\ \hline
Área sector circular   & $A = \displaystyle\frac{\pi * r^{2} * \alpha}{360}$ \\ \hline
Área corona circular   & $A = \pi  (R^{2} - r^{2})$ \\ \hline
\multicolumn{2}{|c|}{} \\
\multicolumn{2}{|c|}{TRIÁNGULO} \\ 
\multicolumn{2}{|c|}{} \\ \hline
\multicolumn{2}{|p{12cm}|}{Considerando $b$ como la longitud de la base, $h$ como la altura, letras minúsculas como la longitud de los lados, letras mayúsculas como los ángulos, y $r$ como el radio de círcunferencias asociadas.} \\ \hline
Área conociendo base y altura & $A = \displaystyle\frac{1}{2}b * h$ \\ \hline
Área conociendo 2 lados y el ángulo que forman & $A = \displaystyle\frac{1}{2}b*a*sin(C)$ \\ \hline
Área conociendo los 3 lados & $ A = \sqrt{p(p - a)(p - b)(p - c)}$ con $p = \displaystyle\frac{a + b + c}{2}$ \\ \hline
Área de un triángulo circunscrito a una circunferencia & $A = \displaystyle\frac{abc}{4r}$ \\ \hline
Área de un triángulo inscrito a una circunferencia & $A = r(\displaystyle\frac{a+b+c}{2})$ \\ \hline
Área de un triangulo equilátero & $A = \displaystyle\frac{\sqrt{3}}{4}a^{2}$ \\ \hline
\multicolumn{2}{|c|}{} \\
\multicolumn{2}{|c|}{RAZONES TRIGONOMÉTRICAS} \\
\multicolumn{2}{|c|}{} \\ \hline
\multicolumn{2}{|p{12cm}|}{Considerando un triangulo rectángulo de lados $a, b$ y $c$, con vértices $A, B$ y $C$ (cada vértice opuesto al lado cuya letra minuscula coincide con el) y un ángulo $\alpha$ con centro en el vertice $A$. a y b son catetos, c es la hipotenusa:}
\\ \hline
\multicolumn{2}{|p{12cm}|}{
$sin(\alpha) = \displaystyle\frac{cateto\ opuesto}{hipotenusa} = \displaystyle\frac{a}{c}$ 

} 
\\ \hline
\multicolumn{2}{|p{12cm}|}{
$cos(\alpha) = \displaystyle\frac{cateto\ adyacente}{hipotenusa} = \frac{b}{c}$ 

} 
\\ \hline
\multicolumn{2}{|p{12cm}|}{
$tan(\alpha) = \displaystyle\frac{cateto\ opuesto}{cateto\ adyacente} = \frac{a}{b}$ 

} 
\\ \hline
\multicolumn{2}{|p{12cm}|}{
$sec(\alpha) = \displaystyle\frac{1}{cos(\alpha)} = \frac{c}{b}$ 

}
\\ \hline
\multicolumn{2}{|p{12cm}|}{
$csc(\alpha) = \displaystyle\frac{1}{sin(\alpha)} = \frac{c}{a}$ 

} 
\\ \hline
\multicolumn{2}{|p{12cm}|}{
$cot(\alpha) = \displaystyle\frac{1}{tan(\alpha)} = \frac{b}{a}$ 

}
\\ \hline
\multicolumn{2}{|c|}{} \\
\multicolumn{2}{|c|}{PROPIEDADES DEL MÓDULO (RESIDUO)} \\
\multicolumn{2}{|c|}{} \\ \hline
Propiedad neutro & (a \% b) \% b = a \% b \\ \hline
Propiedad asociativa en multiplicación &  (ab) \% c = ((a \% c)(b \% c)) \% c \\ \hline
Propiedad asociativa en suma & (a + b) \% c = ((a \% c) + (b \% c)) \% c \\ \hline
\multicolumn{2}{|c|}{} \\
\multicolumn{2}{|c|}{CONSTANTES} \\
\multicolumn{2}{|c|}{} \\ \hline
Pi & $\pi = acos(-1) \approx 3.14159$ \\ \hline
e & $e \approx 2.71828$ \\ \hline
Número áureo & $\phi = \displaystyle\frac{1 + \sqrt{5}}{2} \approx 1.61803$ 

\\ \hline


\end{supertabular}
}
\end{center}